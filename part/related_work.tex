\section{Related work}
Originally, total people in the crowd were count by detection. The method in this group mostly using a sliding window with a classifier trained on low-level hand-craft features, such as HOG [8] and Haar [9], make attempt to detect all human in the scene. The total count is the number of successful detections.

However, detection approaches become challenging when crowd density increases. Counting by regression was proposed to avoid the hard task of object detection. The approach output one single count number for each input image, without detecting every object. Davies et al [12] trained a linear regression on low-level features such as foreground pixels, edges that output a single count number. Idress et al [6] combine multiple methods, including SVM with SIFT feature, HOG based head detection, Fourier analysis. Wang et al [13] were the first to use the Convolution neural network (CNN) in crowd counting problems.

While counting by regression avoids difficulty in detection problem, it does not restrain any location information. Lempitsky et al \cite{lempitsky2010learning} proposed a way to incorporate spatial information into annotated datasets by place a single dot (not bounding-box) at each object of interest and a counting framework to train a regressor that maps input image to density map. Furthermore, the authors argued that dot annotation can be acquired with comparable cost to number-only annotation due to the nature in the way human annotators count a large number of objects. Pham et al \cite{7410729} improved the method by replacing linear regressors with random forest. Further literature focuses on CNN-based approaches due to the great successes of deep learning and CNN-based approaches in the computer vision paradigm. Early CNN-based approaches in density estimation have been carefully surveyed by Sindagi et al \cite{SINDAGI20183}.




C-CNN \cite{9053780} focus on light weight 